\chapter{Obserwacje i wyniki}
Przeprocesowanie dostarczonego zbioru danych zajęło 11 dni, co poskutkowało zebraniem około 30GB informacji na temat analizowanych domen.

\section{Typy odebranych odpowiedzi}
Wśród przeanalizowanych par domen i serwerów można zaobserwować kilka charakterystycznych typów odpowiedzi. Odpowiedzi mogą być dzielone na kategorie ze względu na różne kryteria. Pierwszym kryterium, które będzie brane pod uwagę w niniejszej pracy jest po prostu rozmiar pliku przechowującego informacje pobrane z serwera autorytatywnego. Jest to motywowane faktem, że w istocie im większy jest plik z odpowiedzią, tym spodziewamy się, że zawiera on więcej informacji na temat odpytywanej domeny. 

Pierwszym typem jest odpowiedź, która zajmowała na dysku specyficzną ilość miejsca -- 25123 bajty. Zaobserwowano, że uzyskano 626288 odpowiedzi tego typu. Rozmiar pliku jest swego rodzaju skutkiem ubocznym uproszczonej implementacji skanera. Trudno było przewidzieć wszystkie specyficzne przypadki towarzyszące transferowi strefy DNS. Zaobserwowana sytuacja jest właśnie jednym z tych nietypowych sytuacji i nawet program dig nie odzworowuje idealnie zachowania jakie powinno nastąpić w takiej sytuacji. przyczyna utworzenia opisanego wcześniej pliku nie jest jednoznacznie określona. Zostały podjęte próby ustalenia czym spowodowane jest takie zachowanie. W takich samych przypadkach program dig zwraca jedynie rekord DNS SOA i komunikat o błędzie (ang. \textit{communications error: end of file}). Zachowanie programu dig w dużym stopniu przypomina przekierowanie zapytania IXFR na AXFR (ang. \textit{AXFR fallback}) opisane między innymi w RFC1995\cite{RFC1995}. Wywołanie AXFR fallback następuje w sytuacji, kiedy numer wersji pliku strefy przysłany do serwera jest wyższy niż numer wersji aktualnie na nim przechowywany. Na różnego rodzaju forach\cite{powerdns-forum} czy w serwisach internetowych\cite{powerdns-git}, problem który został opisany pojawia się najczęściej jako problem implementacji oprogramowania PowerDNS\cite{powerdns}. Niemniej jednak nie ustalono dokładnie jaki jest powód wysłania tak dużego pakietu w odpowiedzi na zwykłe zapytanie AXFR.

