\chapter{Obserwacje i wyniki}
Przeanalizowanie dostarczonego zbioru danych zajęło 11 dni.

\section{Typy odebranych odpowiedzi}
Wśród przeanalizowanych par domen i serwerów można zaobserwować kilka charakterystycznych typów odpowiedzi. 

Pierwszym jest odpowiedź, która zajmowała na dysku specyficzną ilość miejsca -- 25123 bajty. Zaobserwowano, że uzyskano 626289 odpowiedzi tego typu. Rozmiar pliku jest swego rodzaju skutkiem ubocznym uproszczonej implementacji skanera. Trudno było przewidzieć wszystkie specyficzne przypadki towarzyszące transferowi strefy DNS. Zaobserwowana sytuacja jest właśnie jednym z tych nietypowych sytuacji i nawet program dig nie odzworowuje idealnie zachowania jakie powinno nastąpić w takiej sytuacji. przyczyna utworzenia opisanego wcześniej pliku nie jest jednoznacznie określona. Zostały podjęte próby ustalenia czym spowodowane jest takie zachowanie. W takich samych przypadkach program dig zwraca jedynie rekord DNS SOA i komunikat o błędzie (ang. \textit{communications error: end of file}). Zachowanie programu dig w dużym stopniu przypomina przekierowanie zapytania IXFR na AXFR (ang. \textit{AXFR fallback}) opisane między innymi w RFC1995\cite{RFC1995}. Wywołanie AXFR fallback następuje w sytuacji, kiedy numer wersji pliku strefy przysłany do serwera jest wyższy niż numer wersji aktualnie na nim przechowywany.s