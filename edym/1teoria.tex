\chapter{Wstęp}

\section{Ataki rekonesansowe}
Ataki rekonesansowe są typem ataków komputerowych których głównym celem jest pozyskanie informacji na temat atakowanego systemu bądź podatności które w nim występują. Słowo ,,rekonesans'' zostało zapożyczone z militarnej nomenklatury oknoszące się do zapoznania z terenem wroga. W kontekście ataków na sieci komputerowe stwierdzeniem określa się analogiczny krok -- działanie przed właściwym atakiem. Sam rekonesans można podzielić dodatkowo na dwie kategorie:
\begin{enumerate}
	\item aktywny atak rekonesansowy,
	\item pasywny atak rekonesansowy.
\end{enumerate}
Atak aktywny odnosi się do działania, gdy atakujący podejmuje akcje przez które może wchodzić w interakcję z systemem, na przykład wysyłanie specjalnie spreparowanych zapytań czy skanowanie portów.

Atak pasywny to tylko i wyłącznie obserwacje działającego systemu. Może to być na przykład podsłuchiwanie ruchu, analiza najczęściej odwiedzanych stron, czy choćby przyglądanie się innym procesom aby opowiednio przyrgotować atak właściwy.

Obie wersje ataków rekonesansowych są również częścią tak zwanego etycznego hakowania (ang. \textit{ethical hacking}). Osoby które tym się zajmują (określani po angielsku jako \textit{white hat}) starają się wytknąć błędy i podatności w systemach przy czym starają się nie ingerować w ich działanie.

Rekonesans DNS jest częścią testu penetracyjnego polegającą na pozyskaniu jak największej ilości informacji na temat badanej domeny. Dane uzyskiwane podczas niego odnoszą się zarówno do serwera DNS jak i wpisów które on przechowuje. Zebrane informacje mogą kompromitować infrastrukturę sieciową firmy nie powodując przy tym generowania zbyt podejrzanego ruchu. Między innymi dlatego ważne jest, aby przywiązywać znaczną uwagę do tego kto i w jaki sposób próbuje łączyć się z serwerami autorytatywnymi odpowiedzialnymi za domenę.


AXFR (\textit{ang. Asynchronous Xfer Full Range}) to mechnizm używany w protokole DNS (\textit{Domain Name System}) do transferowania strefy za którą odpowiada serwer nazw. Głównym przeznaczeniem opisywanego standardu był transfer informacji pomiedzy podstawowym i zapasowanym serwerem przestrzeni nazw. Zasada jego działania jest bardzo prosta -- serwer podrzędny (\textit{ang. slave}) przesyła rządanie AXFR do serwera podstawowego (\textit{ang. primary, master}).

Oczywiste jest, że AXFR został wykorzystywany w celach zupełnie innych niż te, do których go zaprojektowano. Mowa tu o sytuacji, w której serwer główny w żaden sposób nie weryfikuje po swojej stronie źródła takiego zapytania. Prowadzi to do sytuacji, w której każdy, kto jest w stanie utworzyć odpowiedni pakiet TCP może wejść w posiadanie informacji o całej strefie, za którą odpowiada odpytywany serwer DNS. Wspomniane przygotowanie pakietu DNS nie jest specjalnie trudne, ponieważ umożliwia to wiele narzędzi, na przykład dig, wchodzący w skład pakietu bind. 

\section{Domain Name System}
Głównym zadaniem protokołu DNS (\textit{Domain Name System}) jest translacja nazw przyswajalnych dla użytkowników (najczęściej alfanumerycznych) na nazwy sieciowe, czyli adresy IP. 

DNS jest jednym z podstawowych elementów internetu. Z wystawionego przez niego interfejsu korzysta wiele usług sieciowych i innych protokołów. Można traktować go jako bazę danych, która jest rozproszona po wielu lokalizacjach. Ponadto, system powinien oraz cechuje się dużą niezawodnością. W tym przypadku została ona osiągnięta poprzez wprowadzenie dość prostego mechanizmu -- nadmiarowości serwerów nazw. To właśnie dzięki tej redundancji DNS cechuje się wysokim wskaźnikiem niezawodności.

System DNS ma charakterystyczną, hierarchiczną strukturę, którą zaprezentowano na rysunku \ref{hierarchy_dns}. Na rysunku przedstawiono węzeł główny (ang. \textit{root}) reprezentowany jako znak pojedyńczej kropki oraz przykład dwóch typów domen pierwszego poziomu.

\begin{center}
	\includegraphics[scale=1]{image/hierarchy_dns}\label{hierarchy_dns}
\end{center}

Dzięki tej hierarchii systemu możemy powiedzieć o DNS, że charakteryzuje go bardzo dobra skalowalność oraz elastyczność. 

Jeśli chodzi o podział domen pierwszego poziomu ze względu na przynależność organizacyjną, to aktualny stan przedstawiony jest w tabeli poniżej.

\begin{table}[]
	\centering
	\caption{Podział domen najwyższego poziomu ze względu na działalność.}
	\label{my-label}
	\begin{tabular}{|r|p{10.5cm}|}
		\hline
		\textbf{TLD} & \textbf{Opis jednostki} \\
		\hline\hline
		com & Jednostki o działalności komercyjnej (ang. \textit{commercial institutions}) \\
		\hline
		edu & Jednostki edukacyjne (ang. \textit{educational institutions})\\
		\hline
		gov & Instytucje rządowe (ang. \textit{government institutions}) \\
		\hline
		mil & Grupy wojskowe (ang. \textit{military groupos}) \\
		\hline
		net & grupy związane z działaniem sieci (ang. \textit{network support centers}) \\
		\hline
		org & Organizacje nonprofit i inne (ang. \textit{nonprofit organizations}) \\
		\hline
		int & Organizacje międzynarodowe (ang. \textit{international organizations}) \\
		\hline 
	\end{tabular}
\end{table}

Przedstawiony podział nie jest stały. Autorzy zastrzegli, że w przysłości może być on rozszerzony o dodatkowe kategorie.

Podział ze względu na położenie geograficzne jest oczywiście bardziej naturalny i łatwy zarówno do wdrożenia jak i zrozumienia. Każdemu z państw przydzielono dwu bądź trzyliterowy identyfikator, który reprezentują domenę najwyższego poziomu odpowiadającą danemu państwu. Powołując się na informacje przedstawione na grafice \ref{hierarchy_dns} TLD o identyfikatorze \textit{uk} odpowiada domenom utożnasmianym z Wielką Brytnią, a \textit{fr} -- domenom francuskim.

Hierarchia systemu DNS wynika z faktu, że domenami internetowymi każdego poziomu może zarządzać inna organizacja. Odnosi się to zarówno do domeny \textit{root}, jak i domen nawyższego poziomu niezależnie od przynależności grupowej. W Polsce identyfikatorem TLD jest sufiks \textit{pl}, a jednostką odpowiedzialną za nią jest CERT Polska \cite{cert}. Jeśli użytkownik chciałby dołączyć ze swoją siecią do internetu powinien zgłosić do odpowiedniej organizacji taką chęć oraz dostarczyć wszystkich niezbędnych informacji które są wymagane przez zarządcę. 

Bardzo ważnym punktem jest wspomniana w poprzednim akapicie ,,chęć'' dołączenia do internetu. Jeśli uzytkownik czy organizacja chcą używać protokołu DNS jedynie do użytku wewnętrznego, to nie ma rystrykcyjnych ograniczeń co do używanych nazw. Jeśli natomiast oczekuje się wystawienia domen w taki sposób, aby były widoczne z zewnątrz, to należy odpowiednio:
\begin{enumerate}
	\item zarejestrować nazwę domeny,
	\item pozyskać adres IP.
\end{enumerate}

Bardzo trafne jest tu porównanie całego systemu Domain Name System do struktury plików w systemach operacyjnych z rodziny UNIX.  Nazwa domeny bezpośrednio określa jej miejsce w całej przestrzeni nazw, podobnie jak ścieżka bezwzględna pliku określa jego miejsce w całym systemie. Po rejestracji domeny, jest ona dołączana w odpowiednie miejsce w hierarchii. Przykład domeny \textit{google.com} razem z jej poddomenami i odpowiednimi miejscami w herarchii systemu przedstawiono na rysunku \ref{example_domain_tree}.

\begin{center}
	\begin{figure}
	\includegraphics[scale=1]{image/domain_tree}\label{example_domain_tree}
	\caption{Położenie domeny w przestrzeni nazw. \cite{domain_tree_src}}
	\end{figure}
\end{center}

Serwer autorytatywny (ang. \textit{domain authority}) posiada odgórne przyzwolenie na zarządzanie nazwami swoich hostów (kto przyznaje(?)). Ze względu na drzewiastą strukturę systemu można oczekiwać, że kolejne domeny oraz ich serwery autorytatywne będą delegować odpowiedzialność za kolejne strefy do serwerów niższego poziomu. W ten sposób, powołując się na przykład przedstawiony na rysunku \ref{example_domain_tree} serwer autorytatywny com. zarządza nazwami w domenie com, natomiast zarządzanie nazwami w domenie google.com przekazuje do niższego szczeblem serwera przestrzeni nazw. W ten sposób serwer domeny google.com ma możliwość przypisywania nazw takim domenom jak zaprezentowane support.google.com.

\subsection{Domena DNS}
Domeną określa się dany podzbiór herarchii DNS. Są to wszystkie poddomeny podlegające tej samej domenie wyższego poziomu. Odnosząc się do wcześniej przywołanego porównania do systemu plików -- domena to odpowiednik folderu. Może zawierać kolejne domeny, może być określana zarówno na bardzo wysokim (domeny poziomu TLD) jak i bardzo szczegółowym(domeny 2-3 LD) poziomie abstrakcji. Jeśli chcielibyśmy reprezentować system DNS jako drzewo, to domeną DNS nazwiemy węzeł drzewa i wszystkie węzły które są jego potomkami. Pojęcie domeny i elementy które ono określa przedstawiono na rysunku \ref{domain_dns_example}.

\subsection{Strefa DNS}
Strefa DNS jest pojęciem, które określa zbiór domen za które odpowiedzialny jest konkretny serwer przestrzeni nazw. Jeśli wyobrazimy sobie hierarchię DNS jako drzewo, w którym węzłami są kolejne nazwy domen, to strefą DNS będzie zbiór węzłów tego drzewa. Zbiór węzłów wyznaczany jest na podstawie informacji o serwerze autorytatywnym odpowiedzialnym za daną domenę. Te węzły (domeny) których nazwami zarządza ten sam serwer znajdują się w jednej strefie DNS. Przykładowy podział systemu na konkretne strefy został zaprezentowany na rysunku \ref{dns_zone_example}.

\subsection{Strefa DNS a domena DNS}
System DNS danej domeny jest zrealizowany w oparciu o zbiór serwerów przestrzeni nazw (ang. \textit{nameserver}). Każdy z serwerów może być serwerem autorytatywnym dla pojedyńczej domeny, wielu domen bądź domen wraz z odpowiadającymi im poddomenami. Wycinek przestrzeni zarządzany przez określony serwer nazywany jest strefą DNS. 


\subsection{Kompletne nazwy domen}
Określeniem pełne, kompletne lub zupełne nazwy domen (ang \textit{Fully Qualified Domain Names (FQDNs)}) nazywane są te domeny, których nazwy zawierają domenę każdego poziomu, począwszy od lokalnej, aż po domenę główną, czyli root. Łatwo dostrzec analogię do wcześniej wspomnianego systemu pliku w systemach operacyjnych UNIX pomiędzy FQDN a bezpośrednią ścieżką do pliku. Różnicą jest tu jednak sposób odczytywania takiej ścieżki. W systemach DNS najbardziej ogólny węzeł znajduje się na skrajnie prawej pozycji a poruszając się w stronę lewą dochodzimy do kolejnych lokalnych domen. 

\subsection{Mapowanie nazw na adres IP}\label{mapping}
Mapowanie nazw domeny na jej adres IP jest możliwe dzięki plikom strefy DNS, które znajdują się na autorytatywnym serwerze przestrzeni nazw, tzw. \textit{zone files}. Jeden z typów tych plików przechowuje nazwy domen wraz z odpowiadającymi im adresami IP. Gdy klient chce dowiedzieć się pod jaki adres kierować swoje zapytanie, kieruje informacje do serwera autorytatywnego. On odpowiada za znalezienie i przedstawienie mapowania nazwy na adres IP, dokonując przeszukania plików strefy.

\subsection{Mapowanie odwrotne}\label{revmapping}
Baza danych DNS może także zawierać pliki, które umożliwiają mapowania odwrotne, tj. adresu IP na nazwę domeny/hosta. Mechanizm ten może być użyty przy próbie weryfikacji pochodzenia wiadomości. Chcąc wykluczyć próbę oszustwa ze strony prawdziwego nadawcy możemy posiłkować się właśnie odwrotnym mapowaniem. Możemy zweryfikować, czy adres IP mapowania odwrotnego zgadza się z adresem IP z którego nadano wiadomość. Mechanizm ten może być również wykorzystywany do autoryzacji operacji wykonywanych zdalnie. 

Odwrotne mapowanie wykorzystuje specjalną domenę \textit{in-addr.arpa}. Z założenia domena ta wykorzystuje adresy IP zamiast adresów domen.  Wspomniana domena jest częścią strefy DNS, która umożliwia takie właśnie odwzorowanie. Istotne jest, że adresy w domenie \textit{in-addr.arpa} są zapisywane w specyficzny dla siebie sposób -- od najniższego do najbardziej istotnego poziomu. Wynika z tego, że adresy IP w opisywanej domenie są w pewnym sensie zapisywane ,,od końca''. Powołując się na przykład, załóżmy, że maszyna ma przypisany adres 10.8.0.32. W plikach strefy dla domeny in-addr.arpa adres ten będzie zapisany jako 32.0.8.10.in-addr.arpa. oczywiście z kropką po członie \textit{in-addr.arpa}, która reprezentuje domenę \textit{root}.

\subsection{Wplisy plików strefy}
Tak jak wspomniano we wcześniejszych punktach \ref{mapping} oraz \ref{revmapping} działanie całego mechanizmu mapowana (czy to właściwego czy odwrotnego) może zachodzić dzięki serwerom autorytatywnym oraz przechowywanym przez nie plikom strefy DNS. 

Z racji, że system DNS dostarcza administratorom wielu funkcji poza standardowym mapowaniem adresów na IP, wpisy w bazach danych mogą być różnych typów. Najpopularniejsze typy to oczywiście te, z którymi można zetknąć się na codzień, na przykład wpis typu \textit{A} -- tłumaczący nazwę hosta na adres IP w wersji 4, \textit{AAAA} -- mapujący nazwę hosta na adres IP w wersji 6, rekordy \textit{CNAME} służące do aliasów, czy MX -- specyfikujące serwer wymiany wiadomości elektronicznych dla danej domeny. Oprócz wspomnianych podstawowych typów twórcy standardu dla systemu DNS wprowadzili także mniej znane typy. Opis zbioru najbardziej istotnych wpisów określonych w dokumencie RFC 1035 \cite{RFC1035} wraz z ich przeznaczeniem umieszczono w tabeli \ref{typyRekordowDns}.

\begin{table}[]
	\centering
	\caption{Rodzaje rekordów w bazach danych serwerów przestrzeni nazw. Opis na podstawie \cite{}}
	\label{typyRekordowDns}
	\begin{tabular}{|r|p{3cm}|p{8cm}|}
		\hline
			\textbf{Rekord} & 
			\textbf{Pełna nazwa} & 
			\textbf{Pełniona funkcja(opis zawartości)} \\
		\hline\hline
			A & 
			Wpis mapowania adresów(ang. \textit{Address mapping record}) & 
			Określa adres IP wersji 4 dla danego hosta. Służą do konwersji nazwy domen do opowiednich adresów IP.\\
		\hline
			CNAME & 
			Nazwa kanoniczna (ang. \textit{Canonical Name)} & 
			\\
		\hline
			MX & 
			Wymiana wiadomości elektronicznych (ang. \textit{Mail exchange)} & 
			\\
		\hline
			NS & 
			Serwer nazw (ang. \textit{Name server)} & 
			\\		
		\hline
			SOA & 
			Początek (ang. \textit{Start of Authority)} & 
			\\
		\hline
			TXT & 
			Tekst & 
		\\
		\hline
		
	\end{tabular}
\end{table}

Oczywiście na przestrzeni lat protokół rozszerzano o kolejne funkcje. Jednym z istotnych wydarzeń było zaproponowanie zmian spisanych w dokumencie RFC 4034 \cite{}. Opisano tam rozszerzenia protokołu DNS, które podnoszą jego bezpieczeństwo. W tabeli \ref{typyRekordowDnsExt} przedstawiono najważniejsze z punktu widzenia niniejszej pracy typy rekordów, które zostały wprowadzone w kolejnych dokumentach RFC, których dokładne numery zostały wspomniane przy konkretnych wpisach w tabeli. 

\begin{table}[]
	\centering
	\caption{Rodzaje rekordów w bazach danych serwerów przestrzeni nazw określonych w rozszerzeniach dokumentu RFC 1035 \cite{}.}
	\label{typyRekordowDnsExt}
	\begin{tabular}{|r|p{3cm}|p{8cm}|}
		\hline
		\textbf{Rekord} & 
		\textbf{Pełna nazwa} & 
		\textbf{Pełniona funkcja(opis zawartości)} \\
		\hline\hline
			AAAA & 
			Wpis mapowania adresów IPv6 & 
			Określa adres IP wersji 6 dla danego hosta. Zasada działania jest taka jak w przypadku rekordu A z jedyną różnicą w wersji adresu IP. \\
		\hline
	\end{tabular}
\end{table}

\subsection{TLD ARPA}
Specyficzną domeną najwyższego rzędu jest domena \textit{ARPA}.

\subsection{Domena 2LD e164}

\subsection{Domena 2LD in-addr}

\section{Podpisy TSIG}\label{TSIG}
Istotnym typem rekordu jest również typ 250. -- rekord TSIG(ang. \textit{Secret Key Transaction Authentication for DNS/Transaction Signature}) zdefiniowany w dokumencie RFC 2845\cite{RFC2845}. Używany jest przede wszystkim w protokole DNS aby zapewnić, że informacja przesyłana pomiędzy obiema komunikującymi się stronami faktycznie pochodzi od nadawcy oraz że nie była modyfikowana w trakcie komunikacji. System uzywany jest przede wszystkim do dynamicznych aktualizacji baz DNS oraz do transferów stref pomiędzy serwerem głównym a podrzędnym. Aby komunikacja była kryptograficznie bezpieczna, w protokole wykorzystywane są klucze tajne oraz bezkolizyjne funkcje skrótu.

\subsection{Opis działania TSIG}
Rekord TSIG pozwala na wykorzystywanie mechanizmów znanych z protokołu DNSSEC\cite{nask-tsig} w protokole DNS zaproponowanych w RFC 1035\cite{RFC1035}. Praktyka taka została zaproponowana z powodu częstych problemów z używaniem protokołu DNSSEC\cite{RFC4033, RFC4035}. Wprowadzenie korzystania z rekordów TSIG pozwala między innymi na:
\begin{enumerate}
	\item kontrolowanie aktualizacji stref DNS,
	\item zabezpieczenie transferu strefy DNS,
	\item zabezpieczenie komunikacji pomiędzy aktorami (na przykład pomiędzy serwerami przestrzeni nazw).
\end{enumerate}

Zgodnie z nazwą, rekord TSIG jest w głównej mierze kontenerem mającym za zadanie przechowywanie podpisu danej wiadomości DNS. Gdy zostanie wykorzystany jest porzucany zatem nie ma powodu, dla którego rekordy tego typu powinny być przechowywane przechowywane. TSIG znajduje się w części dodatkowej APDU DNS. Po odebraniu pakietu zawierającego TSIG, odpowiedni rekord z sekcji dodatkowej zostaje usunięty i zapisany w oddzielnym miejscu w pamięci. Nagłówek wiadomości jest odpowiednio modyfikowany, tak aby pola dłogości jak i liczby odpowiedzi od serwerów były zgodne z faktycznym stanem po dokonanej modyfikacji. Następnie liczony jest skrót wiadomości z wstępnie przeprocesowanego pakietu i porównywany z podpisem zapisanym uprzednio w innym miejscu w pamięci. W sytuacji, gdy wartość obliczona przez funkcję skrótu jest różna od wartości odebranej w rekordzie TSIG razem z całą odpowiedzią DNS pakiet taki należy odrzucić oraz powiadomić o tym fakcie nadawcę wiadomości. Rekord TSIG niesie także informacje o dwóch czasach: pierwszy -- kiedy utworzono skrót oraz drugi -- jak długo skrót zachowuje swoją ważność. Weryfikacja odebranego pakietu obejmuje nie tylko wartość funkcji skrótu ale także opisany czas. Jeśli czas odebrania wiadomości nie zawiera się w okresie jej ważności odsyłany jest komunikat o błędzie. Dodanie czasu utworzenia pakietu było konieczne, aby zabezpieczyć się przed atakami odtworzeniowymi, gdzie atakujący wykorzystuje podsłuchany pakiet ponownie. Wykorzystywanie stempli czasowych wymaga użycia odpowiedniego zegara. Nie jest to problem jeśli maszyna jest podłączona do internetu, ponieważ może być wykrozystany wtedy protokół NTP (ang. \textit{Network Time Protocol})\cite{RFC5905}.

Generowanie podpisanej odpowiedzi może mieć miejsce tylko po odebraniu podpisanego zapytania od klienta. Serwer nie może wysłać odpowiedzi zawierającej rekord TSIG jeśli otrzymał niepodpisane zapytanie. Generacja skrótu znajdującego się w odpowiedzi składa się zarówno ze skrótu który przysłał klient jak i zawartości rekordu TSIG\cite{nask-tsig}.

Użycie mechanizmu podpisywania wiadomości eliminuje może eliminować problem nieuprawnionego transferu danych. Z pewnością wyklucza użycie algorytmu wykorzystywanego w niniejszej pracy magisterskiej, gdyż łamanie nawet prostych kluczy wielokrotnie zwiększa czas procesowania pojedyńczej domeny\cite{nask-tsig}. Oczywiście zabezpieczenie kluczy używanych do podpisywania wiadomości a także ich długość i jakoś są bardzo ważne w kwestii poziomu bezpieczeństwa protokołu. Zaleca się, aby długość generowanego skrótu była mniejsza bądź równa długości klucza użytego do podpisania wiadomości.

\subsection{Wady TSIG}
Problemem związanym z wykorzystaniem rekordów TSIG jest przede wszystkim dystrybucja kluczy. Ponadto w systemie DNS rzadko zdarza się, że tylko jeden klient będzie korzystał z interfejsu serwera, więc każdy z klientów powinien posiadać swój klucz, co ponownie prowadzi do problemu ich dystrybucji, przechowywania i zarządzania\cite{nask-tsig}. Inną istotną wadą protokołu jest rodzaj zaproonowanej funkcji skrótu HMAC-MD5. Algorytm ten nie jest uważany w dzisiejszych czasach za bezpieczny. Ataki na HMAC-MD5 zaprezentowano między innymi w pracach \cite{hmac-md5-attack, hmac-md5-cryptoanalisys} 

\section{Inne metody podpisywania wiadomości}
Oprócz przedstawionego w podpunkcie \ref{TSIG} mechanimzu TSIG postało kilka innych propozycji podpisywania wiadomości w protokole DNS. Propozycje te opracowaywane były przede wszystkim dlatego, że TSIG charakteryzuje się pewnymi uciążliwymi wadami -- nie rozwiązuje problemu dystrybucji kluczy, nie uwzględnia poziomów w hierarchii systemu DNS czy wykorzystuje przestażałą funkcję skrótu HMAC-MD5. 

\chapter{Przegląd dostępnych narzędzi}
Społeczność internetu oferuje bardzo szeroką gamę narzędzi umożliwiających pozyskiwanie informacji na podstawie protokołu DNS. Bardzo użytecznym i przydatnym narzędziem jest w tym przypadku program dig. Wspomniany program ma jednak znaczącą wadę -- jest wydajny jedynie przy niewielkiej liczbie odpytywanych domen. Dodatkową niedogodnością jest, wbrew pozorom, fakt, że program wchodzi w skład pakietu Open Source bind. Ten, jak każde oprogramowanie na wolnej licencji, cierpi na szereg niedogodności z tym związanych. Najbardziej prozaicznym problemem jest fakt, że oprogramowanie jest tworzone przez wiele osób, więc bardzo trudno zastosować jeden standard kodowania, gdyż każda z osób ma swój preferowany. Poza tym, dużą wadą jest trudność wprowadzania zmian w tego typu oprogramowaniu, wynikająca zarówno z punktu poprzedniego jak i ze złożoności programu, którą cechuje się dig w tym momencie.

Istnieją także inne pakiety implementujące w dość wydajny sposób klienta systemu DNS jak np. pjlib \cite{pjlib}. Również w tym przypadku można borykać się z problemami wynikającymi z założeń przyjętych przez twórców tego oprogramowania. Koknkretyzując to stwierdzenie - wspomniana biblioteka z założenia miała być wykorzystywana do protokołu SIP, a więc implementacja klienta DNS weszła w jej skład tylko i wyłączenie dlatego, że twórcy jej potrzebowali jako narzędzia do zrealizowania innych celów. Implikuje to fakt, że pobieranie wiadomości jest niekompletne na płaszczyźnie typów wiadomości. Jednym z nich jest typ nr 252, czyli AXFR, który jest jednym z kluczowych elementów rekonsesansu na podstawie protokołu DNS.