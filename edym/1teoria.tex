\chapter{Wstęp}
Rekonesans DNS jest częścią testu penetracyjnego polegającą na pozyskaniu jak największej ilości informacji na temat domeny. Dane uzyskiwane podczas niego odnoszą się zarówno do serwera DNS jak i wpisów które on przechowuje. Zebrane informacje mogą kompromitować infrastrukturę sieciową firmy nie powodując przy tym generowania zbyt podejrzanego ruchu. Między innymi dlatego ważne jest, aby przywiązywać znaczną uwagę do tego kto i w jaki sposób próbuje łączyć się z serwerami autorytatywnymi odpowiedzialnymi za domenę.


AXFR (\textit{ang. Asynchronous Xfer Full Range}) to mechnizm używany w protokole DNS (\textit{Domain Name System}) do transferowania strefy za którą odpowiada serwer nazw. Głównym przeznaczeniem opisywanego standardu był transfer informacji pomiedzy podstawowym i zapasowanym serwerem przestrzeni nazw. Zasada jego działania jest bardzo prosta -- serwer podrzędny (\textit{ang. slave}) przesyła rządanie AXFR do serwera podstawowego (\textit{ang. primary, master}).

Oczywiste jest, że AXFR został wykorzystywany w celach zupełnie innych niż te, do których go zaprojektowano. Mowa tu o sytuacji, w której serwer główny w żaden sposób nie weryfikuje po swojej stronie źródła takiego zapytania. Prowadzi to do sytuacji, w której każdy, kto jest w stanie utworzyć odpowiedni pakiet TCP może wejść w posiadanie informacji o całej strefie, za którą odpowiada odpytywany serwer DNS. Wspomniane przygotowanie pakietu DNS nie jest specjalnie trudne, ponieważ umożliwia to narzędzie dig, wchodzące w skład pakietu bind. 

\chapter{Przegląd dostępnych narzędzi}
Społeczność internetu oferuje bardzo szeroką gamę narzędzi umożliwiających pozyskiwanie informacji na podstawie protokołu DNS. Bardzo użytecznym i przydatnym narzędziem jest w tym przypadku program dig. Wspomniany program ma jednak znaczącą wadę -- jest wydajny jedynie przy niewielkiej liczbie odpytywanych domen. Dodatkową niedogodnością jest, wbrew pozorom, fakt, że program wchodzi w skład pakietu Open Source bind. Ten, jak każde oprogramowanie na wolnej licencji, cierpi na szereg niedogodności z tym związanych. Najbardziej prozaicznym problemem jest fakt, że oprogramowanie jest tworzone przez wiele osób, więc bardzo trudno zastosować jeden standard kodowania, gdyż każda z osób ma swój preferowany. Poza tym, dużą wadą jest trudność wprowadzania zmian w tego typu oprogramowaniu, wynikająca zarówno z punktu poprzedniego jak i ze złożoności programu, którą cechuje się dig w tym momencie.

Istnieją także inne pakiety implementujące w dość wydajny sposób klienta systemu DNS jak np. pjlib \cite{pjlib}. Również w tym przypadku można borykać się z problemami wynikającymi z założeń przyjętych przez twórców tego oprogramowania. Koknkretyzując to stwierdzenie - wspomniana biblioteka z założenia miała być wykorzystywana do protokołu SIP, a więc implementacja klienta DNS weszła w jej skład tylko i wyłączenie dlatego, że twórcy jej potrzebowali do innych celów. Implikuje to fakt, że pobieranie wiadomości jest niekompletne na płaszczyźnie typów wiadomości. Jednym z nich jest typ nr 252, czyli AXFR, który jest jednym z kluczowych elementów rekonsesansu na podstawie protokołu DNS.