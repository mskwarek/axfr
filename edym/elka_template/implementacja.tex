\chapter{Duze tabelki}
\lipsum[22-26]
\begin{longtable}{|p{2.5cm}|p{6.5cm}|p{6.5cm}|}
	\hline
	\textbf{Narzędzie} &
	\textbf{Zalety} &
	\textbf{Wady} \\ \hline\hline
	coś &\begin{itemize}
		\item \lipsum[1]
	\end{itemize} &
	\begin{itemize}
		\item \lipsum[4]
	\end{itemize}  \\
	\hline
	inne coś & \lipsum[2] & \lipsum[3]\\
	\hline
	\caption{Podpis długiej tabelki.}
	\label{tab:narzedzia}
\end{longtable}

\section{Jakiś wzór}
\noindent \lipsum[7]:
$$3(s) * 5 * 10^{9} = 25 * 10^{8}(min) = \frac{25}{60} * 10^{8}(h) = 41.67 * 10^{6} (dni)\label{obliczenia}$$

\section{Listowanie}
\begin{itemize}
	\item \lipsum[8]
	\item \lipsum[9]
\end{itemize}

\subsection{Podpunkt}
\label{labelka_dla_podpunktu}
\noindent \lipsum[10-12]
