\chapter{Wstęp}
\noindent \lipsum[2-4].

\section{Rozwinięcie}
\noindent \lipsum[5-6] Odwołanie się do tabelki \ref{tab:tabelka}.

\begin{longtable}{|c|p{4cm}|p{8cm}|}
	\hline
	\textbf{Lp.} &
	\textbf{A} &
	\textbf{B} \\ \hline\hline
	1 & C & \lipsum[7]\\
	\hline
	\caption{Opis na podstawie \cite{domain_tree_src}.}
	\label{tab:tabelka}
\end{longtable}

\lipsum[10-12] Odwołanie się do obrazka \ref{obrazek}.

\begin{center}
	\begin{figure}
	\centering
	\includegraphics[scale=0.3]{image/obrazek}
	\caption{Podpis}
	\label{obrazek}
	\end{figure}
\end{center}

\subsection{Kolejna sekcja}
\noindent \lipsum[12-13] Ponowne odwołanie się do listingu bo nie chciało mi się go usunąć \ref{list:listing}.

\begin{lstlisting}[label={list:listing},captionpos=b,caption=Przykładowy listing.,language=c]
	#include <iostream>

	int main()
	{
	  std::cout << "Hello World!";
	  std::cout << std::endl;

	  return 0;
	}
\end{lstlisting}
