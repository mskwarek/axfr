\chapter{Podsumowanie}
\noindent W niniejszej pracy magisterskiej zaproponowano unikatowe podejście do problemu transferu strefy AXFR oparte na globalnym skanowaniu
domen i ich serwerów przestrzeni nazw. Przeanalizowano, czy powszechnie dostępne narzędzia zapewniają takie funkcje, aby przeprowadzić
badania w zakresie postawionego problemu. Została zaproponowana implementacja oraz wdrożenie systemu, który umożliwia optymalne wykonanie
takich badań. Zebrano oraz przeanalizowano dane, które można uzyskać w wyniku globalnego skanowania domen pod kątem podatności
na transfer AXFR. Pogrupowano uzyskane odpowiedzi, typy zachowań na wystosowane żądanie AXFR. Opisano, w jaki sposób mogą odpowiadać
serwery autorytatywne oraz rzeanalizowano popoularność każdego z typów otrzymywanej odpowiedzi.

Podsumowując przeprowadzone prace i badania okazuje się, że transfer strefy wykorzystując mechanizm AXFR jest wciąż możliwy dla
bardzo wielu domen. Bardzo duża część domen jest pod zarządem niewielkiej liczby serwerów autorytatywnych, więc możliwe, że zezwolenie
na taki transfer bywa zamierzone. Resztę można zaliczyć jako niepoprawnie skonfigurowane i to one są najbardziej narażone na
ataki cyberprzestępców. Zalecenia wydane w dalszej części rozdziału są kierowane głównie w kierunku administratorów właśnie tych domen.

W pracy opisano także, które części stref DNS pozwalają cyberprzestępcom na pozyskanie informacji na temat ich potencjalnych celów.
Skupiono się w głównej mierze na usługach uruchamianych w badanych systemach, informacjach na temat infrastruktury antyspamowej,
ale także na pośrednim wykorzystaniu danych maszyn w innych typach ataków.

\section{Zalecenia}
\noindent Transfer strefy AXFR pozwala na uzyskanie wielu informacji na temat tego, jakie usługi oraz jakie maszyny są uruchomione wewnątrz danej
domeny. Drastycznym, ale najbardziej skutecznym zaleceniem jest wycofanie z implementacji serwerów DNS funkcji transferu strefy DNS do
każdej maszyny wystosowującej zapytanie tego typu. Wymagałoby to używania jedynie adresów IP serwerów, które mogą ten transfer przeprowadzić.
Rozwiązanie to ograniczyłoby skalowalność rozwiązania, jednak jawne podanie adresu IP jest dużo bardziej bezpieczne i zapewnia, że
transfer strefy przeprowadzony będzie tylko przez te maszyny, którym rzeczywiście, wprost, na to zezwolono.

Innym rozwiązaniem jest rezygnacja z własnych serwerów DNS na rzecz usług \textit{DNS as a service} (rozdział \ref{dnsasservice}). Takie postępowanie eliminuje więszkość
problemów związanych z transferem strefy, gdyż cała infrastruktura jest dzierżawiona od zewnętrznej firmy. Firmom takim również zależy
na jak najlepszej jakości ich usług oraz zależy im, aby dane na temat stref DNS nie były tak powszechnie dostępne, dlatego dobrze
zabezpieczają transfer strefy. Odbywa się to na przykład dzięki zestawieniu sesji SSL pomiędzy stronami wymieniającymi dane, bądź
innym wykorzystaniu kryptografii (na przykład szyfrowanie).

\section{Retrospekcja}
\noindent Pośrednim wynikiem tworzenia niniejszej pracy magisterskiej mogą być również wnioski na temat samego procesu jej realizacji czy przeprowadzania badań.
Nie wszystkie etapy pracy były przeprowadzane płynnie, co pozwala na wyciągnięcie wniosków pozwalających uniknąć takich sytuacji
w przyszłości. Jednym z trudniejszych etapów realizacji pracy było przygotowanie oprogramowania i środowiska pod globalne skanowanie.
Trud ten wynika głównie ze skali, skrypty dostępne w Internecie nie są aż tak wydajne aby zapewnić oczekiwaną wydajność. Poważnym
błędem, który spowodował wydłużenie czasu uruchomienia systemu były próby wykorzystania już istniejących narzędzi, jednak pozwoliło to
na opisanie ich wad bądź zalet. Dodatkowo, można wyciągnąć wniosek, że przeprowadzanie skanowania na szeroką skalę bardzo często wymaga
implementacji oprogramowania dokładnie pod kątem tego specyficznego przypadku, bardzo często bez używania innych projektów powiązanych
z tematyką badań. Oprogramowanie skanera jest ciągle ulepszane, co dowodzi temu, że nawet kilka miesięcy działania systemu nie pozwala
na pełne, dokładne przetestowanie skanera.
