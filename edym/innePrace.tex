\chapter{Powiązane prace}

\section{DNS Response Policy Zone}

\section{Internet-Wide Scan Data Repository}
\textit{Internet-Wide Scan Data Repository} jest to publiczne archiwum danych dostępne głównie pod adresem scans.io\cite{scansio}. Gromadzone są tam dane badawcze zebrane podczas aktywnych skanów przeprowadzonych w internecie. Repozytorium prowadzone jest przez badaczy z Uniwersytetu z Michigan\cite{teamcensys}. Cel pobierania danych nie jest wyraźnie określony przez naukowców, to co zostanie z nich wywnioskowane pozostawia się osobom trzecim. Z punktu widzenia niniejszej pracy, \textit{Internet-Wide Scan Data Repository} może posłużyć jako punkt odniesienia na przestrzeni skanowania AXFR. Na stronie możemy znaleźć wyniki aktywnego skanowania domen z listy Alexa Top 1 Milion\cite{alexa} pod kątem podatności na nieuprawniony transfer strefy DNS.

Jednym z udogodnień ze strony pracowników Uniwersytetu z Michigan jest udostępnienie wygodnego interfejsu dla ludzi. Zarządcy opisywanego repozytorium danych udostępniają również aplikację, wyszukiwarkę informacji na temat domen przeskanowanych pod różnym kątem w ich badaniach. Cały projekt nosi tę samą nazwę co zespół -- Censys\cite{censys} i pozwala wyszukiwać informacje zarówno po adresach URL/IP jak i po zawartości odpowiedzi uzyskanych ze skanów.

Różnicą która jest pomiędzy pracami opisanymi w tym punkcie a niniejszą pracą magisterką jest oczywiście zasięg prowadzonych badań. Ograniczenie się jedynie do miliona najpopularniejszych domen, tak jak to zrobił zespół Censys\cite{censys} jest bardoz dużym uproszczeniem. Ponadto, naturalne jest, że najpopularniejsze strony będą przykłądały dużo uwagi do odpowiedniego zabezpieczenia swoich zasobów, dlatego można się spodziewać, że wyników takiego skanowania będzie mniej niż na równej ilościowo, losowo wybranej próbie. 