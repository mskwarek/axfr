\chapter{Powiązane prace}
Pomimo faktu, że problem ,,wycieku'' danych z serwerów DNS jest dość powszechnie poruszany zarówno w różnego rodzaju publikacjach\cite{} jak i na blogach dotyczących testowania infrastruktury pod kątem bezpieczeństwa\cite{}, to nie udało dotrzeć się do żadnych globalnych badań na temat bezpieczeństwa transferu strefy DNS. To właśnie skala badania przeprowadzonego w ramach niniejszej pracy magisterskiej jest główną różnicą w stosunku do badań już przeprowadzonych. Ponadto, bardzo ważnym założeniem przedstawionym w tej pracy jest chęć nakreślenia skali zjawiska nieuprawnionego transferu danych DNS. Określenie jak duży jest to problem jest konieczne do wydania ewentualnych zaleceń oraz przyporządkowania priorytetu bądź rangi tej podatności. W poniższym rozdziale przedstawiono powiązane prace oraz ich różnice w stosunku do opisywanych badań.

\section{DNS Response Policy Zone}


\section{Internet-Wide Scan Data Repository}
\textit{Internet-Wide Scan Data Repository} jest to publiczne archiwum danych dostępne głównie pod adresem scans.io\cite{scansio}. Gromadzone są tam dane badawcze zebrane podczas aktywnych skanów przeprowadzonych w internecie. Repozytorium prowadzone jest przez badaczy z Uniwersytetu z Michigan\cite{teamcensys}. Cel pobierania danych nie jest wyraźnie określony przez naukowców, to co zostanie z nich wywnioskowane pozostawia się osobom trzecim. Z punktu widzenia niniejszej pracy, \textit{Internet-Wide Scan Data Repository} może posłużyć jako punkt odniesienia na przestrzeni skanowania AXFR. Na stronie możemy znaleźć wyniki aktywnego skanowania domen z listy Alexa Top 1 Milion\cite{alexa} pod kątem podatności na nieuprawniony transfer strefy DNS.

Jednym z udogodnień ze strony pracowników Uniwersytetu z Michigan jest udostępnienie wygodnego interfejsu dla ludzi. Zarządcy opisywanego repozytorium danych udostępniają również aplikację, wyszukiwarkę informacji na temat domen przeskanowanych pod różnym kątem w ich badaniach. Cały projekt nosi tę samą nazwę co zespół -- Censys\cite{censys} i pozwala wyszukiwać informacje zarówno po adresach URL/IP jak i po zawartości odpowiedzi uzyskanych ze skanów.

Różnicą która jest pomiędzy pracami opisanymi w tym punkcie a niniejszą pracą magisterką jest oczywiście zasięg prowadzonych badań. Ograniczenie się jedynie do miliona najpopularniejszych domen, tak jak to zrobił zespół Censys\cite{censys} jest bardzo dużym uproszczeniem. Ponadto, naturalne jest, że najpopularniejsze strony będą przykładały dużo uwagi do odpowiedniego zabezpieczenia swoich zasobów, dlatego można się spodziewać, że wyników takiego skanowania będzie mniej niż na równej ilościowo, losowo wybranej próbie. 

\section{DNS as a service}
DNS as service to pojęcie wykorzystywane do opisywania usług świadczonych w obrębie protokołu DNS. Dosłownym tłumaczeniem tego terminu jest \textit{DNS jako usługa}. W rozumieniu protokołu DNS jest to ograniczone do wystawienia interfejsów klientowi, tak, aby mógł on dodać interesujące go rekordy. 

\textit{DNS as a service} spotykane jest najczęściej w kontekście unikania ataków DDoS jednak mimo to można przyjrzeć się temu zjawisku również przy okazji analizy bezpieczeństwa transferu strefy DNS. Określenie \textit{Distributed Denial of Service} jest w tym przypadku swego rodzaju chwytem marketingowym, ponieważ najbardziej efektywnie działa na wyobraźnię klientów. Niemniej jednak, bezpieczeństwo systemów DNS to nie tylko unikanie ataków DDoS. 

Tym, co faktycznie sprzedają firmy oferujące usługi \textit{DNS as a service} jest przede wszystkim infrastruktura, ale także pewnego rodzaju bezpieczeństwo informacji oraz pewność dostępności systemu DNS. Dobrym przykładem jest w tym przypadku oferta firmy Nexusguard\cite{nexusguard}. Oferuje ona obsługę oraz zabezpieczenie systemu DNS swojego klienta. Wprowadzenie danych na temat strefy DNS klienta może odbywać się na dwa sposoby:
\begin{enumerate}
	\item poprzez panel klienta,
	\item poprzez transfer strefy DNS z serwera klienta.
\end{enumerate}
W kontekście tej pracy magisterskiej pierwszy przypadek nie jest nadzwyczaj interesujący. Klient uzupełnia dane o rekordach DNS swojej domeny czy też strefy i cała odpowiedzialność za rozwiązywanie kolejnych zapytań jest przeniesiona na maszyny działające pod nadzorem firmy oferującej usługę. Dużo bardziej interesującym przypadkiem jest wymieniony jako drugi transfer strefy z serwera klienta. Wydzielona część infrastruktury firmy jest wtedy traktowana jako zapasowy serwer DNS(\textit{ang. slave}) podczas gdy serwerem podstawowym (\textit{ang. primary}) jest maszyna klienta. Transfer strefy DNS jest dużo bardziej bezpieczny. Wynika to z wymagań narzuconych przez organizację, mowa między innymi o mechanizmie TSIG (\textit{ang. Transaction Signature})  opisanym w rozdziale \ref{} oraz wykorzystaniu połączenia szyfrowanego opartego na SSL pomiędzy serwerem podstawowym a podrzędnym. Jeśli chodzi o bezpieczeństwo dostępu do rekordów DNS poprzez panel kliencki, to możliwe jest wykorzystanie dwuetapowej weryfikacji użytkownika.

Warto dodać, że \textit{DNS as a service} jest również w ofercie kilku innych, na przykład Akamai\cite{akamai}, CloudFlare\cite{cloudflare} czy Imperva\cite{incapsula}. Obecność usług takich jak DNS as service sugeruje, że jest to bardzo istotne, aby odpowiednio zabezpieczać nie tylko główne maszyny i usługi w swojej sieci ale i niemal każdy element, który działa w naszym systemie. Dodatkowo sugeruje to, że często administratorzy czy klienci wolą przenieść odpowiedzialność za utrzymywanie takiego systemu na firmę trzecią w zamian za odpowiednie ich wynagrodzenie.

\section{DNS Enumeration}\label{zone_enumeration}
Rozszerzenie DNSSEC \textit{DNS Security Extentions} zostało zaprezentowane w roku 1997\cite{RFC2065} jako odpowiedź na poważne luki w bezpieczeństwie systemów DNS opisane w dokumencie \cite{DNSsecurityProblem}. W dużym uproszczeniu rozszerzenie to pozwala na zweryfikowanie, czy odpowiedź na zapytanie wysłane do serwera pochodzi na pewno od niego a nie od atakującego system. Wyspecyfikowanie standardu DNSSEC było problematyczne, a organizacje pracujące nad nim wielokrotnie go zmieniały. Podyktowane to było różnego rodzaju ograniczeniami technicznymi. Jednym z takich ograniczeń technicznych jest zjawisko \textit{DNS Enumeration} (lub \textit{Zone Enumeration}). Jest to podatność w sieci polegająca na wypisaniu wszystkich urządzeń, które do tej sieci należą. W standardzie DNSSEC umożliwiono \textit{zone enumeration}, którego wynikiem były nawet urządzenia sieciowe (na przykład rutery) znajdujące się w danej strefie. Zjawisko nie jest luką krytyczną, jednak może prowadzić do bardziej złożonych ataków. 

Problem listowania strefy DNS wynika z różnego traktowania odpowiedzi pozytywnej oraz negatywnej od serwera. Odpowiedź pozytywna jest wysyłana w przypadku, gdy dana domena istnieje oraz przypisano jej adres IP. Przykładowo:
\begin{enumerate}
	\item Klient: Jaki jest adres domeny www.przyklad.pl?
	\item Serwer: Domena www.przyklad.pl ma adres 1.2.3.4.
\end{enumerate} 
Jeśli chodzi o odpowiedź negatywną, to przykład podano poniżej:
\begin{enumerate}
	\item Klient: Jaki jest adres dla domeny nieistnieje.przyklad.pl?
	\item Serwer: Domena nieistnieje.przykład.pl nie istnieje.
\end{enumerate}

Oba przytoczone powyżej przypadki są w pewnym stopniu inaczej traktowane w protokole DNSSEC. Dla przykładu, gdy odpowiedzią jest adres IP istniejącej domeny, serwer autorytatywny przechowuje określony, skończony zbiór podpisanych rekordów. Podpisy są tworzone przy użyciu klucza prywatnego danej domeny. Istotny jest fakt, że podpisy rekordów nie są obliczane w czasie rzeczywistym, a jedynie przechowywane razem z innymi rekordami w bazie danych DNS. Zaletą takiego rozwiązania jest oczywiście redukcja obliczeń wykonywanych przez serwer autorytatywny a poza tym, tylko jeden z serwerów musi być w posiadaniu klucza prywatnego, więc dystrybucja klucza jest w dużym stopniu uproszczona. Dodatkowo w przedstawionym wcześniej modelu nie ma potrzeby, aby weryfikować tożsamość każdego z serwerów w systemie DNS, co ponownie upraszcza komunikację, zmniejsza obciążenie w sieci oraz ilość informacji.

Problem listowania strefy DNS pojawia się wraz z charakterystyczną, negatywną odpowiedzią (w specyfikacji określaną jako NXDOMAIN). Nie jest możliwe, aby odpowiadać na pytanie o niepoprawną domenę wcześniej przygotowaną wiadomością z obliczonym skrótem, ponieważ może to prowadzić do skutecznego ataku odtworzeniowego (ang. \textit{reply attack}). Nie jest możliwe obliczenie wszystkich skrótów, dla każdej z poddomen, bo takich przypadków jest zbyt wiele. Ciężko także zrezygnować z niewątpliwej zalety przytoczonej w poprzednim akapicie -- braku konieczności obliczania skrótów w czasie rzeczywistym. Pomysłodawcy rozwiązania opisanego w RFC4034\cite{RFC4034} zaproponowali algorytm, dzięki któremu możliwe będzie utrzymanie substytutu podpisanej wiadomości NXDOMAIN. Cały system zachowuje również zalety przedstawione w poprzednim akapicie. Rozwiązanie zakłada, że rekordy w strefie są uporządkowane, każda para rekordów jest podpisana i tworzy rekord NSEC a każdy rekord podpisany kluczem prywatnym. Jeśli serwer odbierze zapytanie o domenę, która nie istnieje, odpowiada rekordem NSEC pary domen które w uporządkowanej liście znajdują się odpowiednio przed i po odpytywanej domenie oraz związane z nimi podpisy. Rozwiązanie zapewnia, że tylko serwer autorytatywny powinien być obdarzony zaufaniem oraz umożliwia wstępną generację podpisów. Niestety wprowadza dodatkowo efekt uboczny, w postaci umożliwienia \textit{zone enumeration}, czyli wylistowania strefy.

Zagadnienie listowania stref DNS samo w sobie często było podmiotem dyskusji. Początkowo wydano dokument, że nie jest to błędem, że umożliwia się wypisanie bazy danych DNS\cite{RFC4033}. Później jednak wypracowany został kompromis\cite{RFC5155}, że w pewnych przypadkach znajomość wszystkich domen może powodować dodatkowe niebezpieczeństwa. Przykłady, które przytoczono w RFC5155\cite{RFC5155} to na przykład dobre źródło danych wejściowych, które mogą posłużyć jako prawdopodobne adresy mailowe w kampaniach spamowych bądź jako informacje o infrastrukturze wykorzystywane podczas rekonesansu DNS. Ponadto podatność nazywana \textit{zone enumeration} może afektować na organizacje zajmujące się rejestrami DNS. Często są one zobowiązane do nieujawniania danych przechowywanych w swoich rejestrach. Rekord NSEC umożliwia przeprowadzenie \textit{DNS enumeration} na strefach tych organizacji a następnie wykonanie zapytań whois\cite{RFC3912} w celu pozyskania informacji o osobie rejestrującej domenę. 