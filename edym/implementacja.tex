\chapter{Implementacja}
Przeanalizowanie problemu postawionego w niniejszej pracy magisterskiej wymagało implementacji własnych narzędzi. Jak wcześniej wspomniano, programy dostępne w sieci są bardzo dobre (dig \cite{dig}, axfr-tool \cite{git_axfr_tool}, robotex \cite{robotex}) jednak nie zapewniają oczekiwanej wydajności. Wynika to z faktu, że są one nastawione głównie na analizę konkretnego przypadku. Narzędzia są używane głównie przez pentesterów, których typowym zadaniem jest przetestowanie pod kątem bezpieczeństwa pewnej aplikacji. Proces ten sam w sobie jest niezwykle złożony i często wymaga nawet tygodni pracy, dlatego czas w jakim wykonywał się będzie program sprawdzający podatności DNS nie jest aż tak istotny. W podejściu prezentowanym w tej pracy założono, że przeanalizuje się możliwie jak najwięcej domen dostępnych w sieci internet. Dlatego też różnica procesowania jednej pary domena -- adres IP serwera autorytatywnego już nawet na poziomie kilku sekund pozwala zaobserwować bardzo duży skok wydajności. 

Dojście do rozwiązania optymalnego nie było zbyt oczywiste. Początkowe rozwiązania opierały się na próbach nabudowania managera zarządzającego procesami w których uruchamiane były zewnętrzne narzędzia -- głównie dig. Rozwiązanie takie miało szereg zalet, na przykład było bardzo skalowalne i łatwe do zrealizowania. Niestety okazało się, że manager zarządzający procesami wymaga zbyt dużej ilości zasobów, a głównie pamięci RAM, dlatego wymuszone zostało zrezygnowanie z tego podejścia. 

Ostatecznie zdecydowano się na napisanie własnego narzędzia umożliwiającego transfer strefy DNS. Wybór padł na język programowania C \cite{clanguage} w wersji standardu c03 \cite{c03std}. Oczywiście aplikacje pisane w języku C nie należą do rozwiązań prostych, jednak są bardzo dobrym kompromisem pomiędzy wysokim poziomem abstrakcji modeli programistycznych i łatwością dostępu do interfejsów sieciowych. Ponadto język C charakteryzuje się dość dobrą szybkością działania. Kryterium to było jednym z kluczowych przy implementacji.

Rolę danych wejściowych programu mogą pełnić odpowiednio:
\begin{enumerate}
	\item plik tekstowy w formacie \textit{adres\_domeny}|\textit{IP\_serwera\_autorytatywnego},
	\item para parametrów wejściowych podanych w odpowiedniej kolejności - \textit{adres\_domeny IP\_serwera\_autorytatywnego}. 
\end{enumerate}
Zdecydowano się na takie rozróżnienie z prostego powodu. Autor rozwiązania chciał zachować podobieństwo tworzonego rozwiązania do programu dig, który jest wykorzystywany przez wiele osób. Dodatkowo, dużo łatwiej można testować logikę programu na małych porcjach danych i opcja numer 2. została w pewnym sensie funckją debugową. Jako wspomniane dane testowe najczęściej wykorzystywana była domena \textit{zonetransfer.me}\cite{zonetransfer}, której autor specjalnie umożliwia transfer strefy, aby uzmysłowić innym ludziom zagrożenie wynikające z niepowołanego korzystania.

Logika programu jest bardzo prosta. Początkowo budowana jest jednostka APDU protokołu DNS z odpowiednimi danymi, a w szczególności z poprawnie ustawioną flagą \textit{qtype} odpowiedzialną za identyfikację rodzaju zapytania DNS. Co ważne, mechanizm AXFR wymaga używania nietypowego dla DNS protokołu TCP.

Komunikacja z serwerami autorytatywnymi została oparta na blokujących socketach. Dodatkowo, zaimplementowany został autorski mechanizm pozwalający na wykorzystywanie timeoutów podczas łączenia się z serwerem. 

Algorytm postępowania w przypadku odebrania pakietu danych przedstawiony jest na rysunku\ref{}.