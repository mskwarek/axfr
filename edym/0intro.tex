
% \begin{titlepage}
% \begin{center}
% 	\includegraphics[scale=0.33]{image/weiti}
% 	%\setmainfont[Ligatures=TeX]{Helvetica}
% 	\vspace{0.7cm}
% 	\begingroup
% 	\fontsize{12pt}{12pt}\selectfont
% 		\begin{center}
% 			Instytut Telekomunikacji
% 		\end{center}
% 	\endgroup
% 	\vspace{0.7cm}
% 	\includegraphics[scale=0.8]{image/tytul}\\
% 	\vspace{0.3cm}
% 	na kierunku telekomunikacja\\
% 	na specjalności telekomunikacja.\\
% 	\vspace{1.2cm}
% 	\begingroup
% 	\fontsize{14pt}{12pt}\selectfont
% 	\begin{center}
% 	 	\{Tytuł\}\\
% 	\end{center}
% 	\endgroup
% 	\vspace{2.3cm}
%
%
% 	\begingroup
% 	\fontsize{21pt}{12pt}\selectfont
% 	\begin{center}
% 		Marcin Skwarek
% 	\end{center}
% 	\endgroup
%
% 		\begingroup
% 	\fontsize{12pt}{12pt}\selectfont
% 	\begin{center}
% 		numer albumu 257319\\
% 		\vspace{0.9cm}
% 		opiekun\\
% 		dr hab. inż Wojciech Mazurczyk\\
% 		\vspace{6.3cm}
% 		Warszawa 2017
% 	\end{center}
% 	\endgroup
% \end{center}
% \end{titlepage}

\begin{center}
	\fontsize{18pt}{12pt}\selectfont\textbf{Rekonesans DNS na podstawie analizy zawartości zapytań AXFR}\\
	\vspace{1cm}
	\fontsize{15pt}{12pt}\selectfont
	\textbf{Streszczenie}
\end{center}
Celem niniejszej pracy jest omówienie konsekwencji lekceważenia zabezpieczania protokołu DNS, które może powodować poważne incydenty
bezpieczeństwa sieciowego.
Protokół Domain Name System jest jednym z kluczowych elementów sieci Internet, bez którego niemożliwe byłoby jej działanie.
Z racji na swoją istotną rolę, powinien być on jak najlepiej zabezpieczony przed różnego rodzaju atakami oraz być niezawodnym.
Niezawodność została zapewniona między innymi poprzez wprowadzenie redundancji serwerów przestrzeni nazw. Bardzo ważne jest również zapewnienie, że informacje przesyłane
protokołem DNS są spójne oraz autentyczne. Należy także pamiętać o zachowaniu ich prywatności. Informacje na temat infrastruktury
oraz usług przechowywane są w strefach DNS. System DNS został usprawniony dzięki wprowadzeniu mechanizmu AXFR, który umożliwia
transfer strefy z jednego serwera na drugi. Wynikiem transferu tego typu, jest otrzymanie wpisów przechowywanych w danej strefie.
Nieumiejętna konfiguracja serwera DNS może prowadzić do tego, że każda maszyna będzie mogła skutecznie pobrać strefę DNS danego
serwera.
Nieumyślne ujawnianie pozornie nieszkodliwych informacji, może prowadzić do poważnych problemów związanych z bezpieczeństwem sieciowym.
W pracy przeanalizowane zostały różnice oraz podobieństwa serwerów, które umożliwiają przeprowadzenie takiego transferu. Ponadto,
przedstawione zostały korzyści, które może czerpać cyberprzestępca w wyniku pozyskania informacji o strefach DNS.
Cel został osiągnięty poprzez realizację globalnego skanowania domen pod kątem transferu AXFR oraz późniejszej syntezie wyników.\\
\vspace{1cm}

\noindent\textbf{Słowa kluczowe:} \textit{DNS, rekonesans, cyberbezpieczeństwo, AXFR}\\
\newpage
\null
\newpage
\begin{center}
	\fontsize{18pt}{12pt}\selectfont\textbf{DNS reconnaisance based on content of AXFR queries analysis}\\
	\vspace{1cm}
	\fontsize{14pt}{12pt}\selectfont
	\textbf{Abstract}
\end{center}
The purpose of the thesis is to present consequences of disregard for DNS data protection. Domain Name System protocol is one of
crucial elements on the Internet that run network. Due to its significant role, it has to be properly secured against various
attacks and reliable. The latter is achieved through implementation of redundant namespace server. Furthermore, information sent
through DNS should be coherent and authentic, keeping in mind its privacy. Infrastructure and services information is stored in DNS zones.
DNS was streamlined by introducing AXFR protocol that defines zone transfer from one server to another. The result of that type
of transfer is obtaining entry in certain zone. Improper DNS server configuration can lead to the situation that any computer
will be able to download zone files successfully.
Unintentional disclosure of potentially harmless information may lead to severe network security problems.
Thesis involves analysis of differences and similarities between servers, which enables carrying out transfer operations.
There are also presented potential benefits for cybercriminal gaining information about DNS zones.
The aim may be achieved by global domain scanning, paying special attention to AXFR transfer and further synthesis of results.\\
\vspace{1cm}

\noindent\textbf{Keywords:} \textit{DNS, reconnaissance, cybersecurity, AXFR}\\
\vspace{1.5cm}

\newpage
\null
\newpage
