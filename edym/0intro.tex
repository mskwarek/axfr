
\begin{titlepage}
\begin{center}
	\includegraphics[scale=0.33]{image/weiti}
	%\setmainfont[Ligatures=TeX]{Helvetica}
	\vspace{0.7cm}
	\begingroup
	\fontsize{12pt}{12pt}\selectfont
		\begin{center}
			Instytut Telekomunikacji
		\end{center}
	\endgroup
	\vspace{0.7cm}
	\includegraphics[scale=0.8]{image/tytul}\\
	\vspace{0.3cm}
	na kierunku telekomunikacja\\
	na specjalności telekomunikacja.\\
	\vspace{1.2cm}
	\begingroup
	\fontsize{14pt}{12pt}\selectfont
	\begin{center}
	 	\{Tytuł\}\\
	\end{center}
	\endgroup
	\vspace{2.3cm}


	\begingroup
	\fontsize{21pt}{12pt}\selectfont
	\begin{center}
		Marcin Skwarek
	\end{center}
	\endgroup

		\begingroup
	\fontsize{12pt}{12pt}\selectfont
	\begin{center}
		numer albumu 257319\\
		\vspace{0.9cm}
		opiekun\\
		dr hab. inż Wojciech Mazurczyk\\
		\vspace{6.3cm}
		Warszawa 2017
	\end{center}
	\endgroup
\end{center}
\end{titlepage}

\newpage

\clearpage
\pagenumbering{gobble}
\newpage
\clearpage
\pagenumbering{gobble}
\begin{center}
	\fontsize{18pt}{12pt}\selectfont\textbf{\{Tytuł\}}\\
	\vspace{1cm}
	\fontsize{14pt}{12pt}\selectfont
	\textbf{Streszczenie}
\end{center}
Protokół Doman Name system jest jednym z kluczowych elementów sieci Internet, bez którego niemożliwe byłoby jej działanie.
Z racji na swoją istotną rolę, powinien być on jak najlepiej zabezpieczony przed różnego rodzaju atakami. Bardzo istotne jest zapewnienie,
aby informacje przesyłane protokołem DNS były spójne oraz autentyczne jednak należy pamiętać, że powinny być one możliwie najbardziej
prywatne. Nieumyślne ujawnianie pozornie nieszkodliwych informacji, może prowadzić do poważnych problemów związanych z bezpieczeństwem
sieciowym. Niniejsza praca przedstawia i omawia przypadki, w których lekceważenie zabezpieczania protokołu DNS może powodować poważne
incydenty bezpieczeństwa sieciowego.\\
\noindent\textbf{Słowa kluczowe:} \textit{DNS, rekonesans, cyberbezpieczeństwo, AXFR}\\
\vspace{1.5cm}

\begin{center}
	\fontsize{18pt}{12pt}\selectfont\textbf{\{Title\}}\\
	\vspace{1cm}
	\fontsize{14pt}{12pt}\selectfont
	\textbf{Abstract}
\end{center}
Domain Name System protocol is one of the key elements in the Internet, which make this network working. Due to that, it have to be
properly secured against various attacks. Besides, information sent through DNS should be coherent and authentic it has to be
known that it should be as private as possible. Serious network security problem could be caused by unintentional disclosure of
potentially harmless information. Therefore, consequences of disregard for DNS data protection was discussed and presented in the thesis.
\\
\noindent\textbf{Keywords:} \textit{DNS, reconnaissance, cybersecurity, AXFR}\\
\vspace{1.5cm}

\newpage
\begin{center}
	\textbf{Życiorys}
\end{center}
\vspace{1cm}
Nazywam się Marcin Skwarek, urodziłem się 10.10.1993 roku w Siedlcach. W 2012 roku ukończyłem I Liceum Ogólnokształcące w Siedlcach
w klasie o profilu matematyczno-fizyczno-informatycznym, w 2013 rozpocząłem studia na Wydziale Elektroniki i Technik Informacyjnych
Politechniki Warszawskiej na kierunku telekomunikacja, które ukończyłem w roku 2016 uzyskując tytuł inżyniera.
