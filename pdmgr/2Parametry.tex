\section{Podobne prace}

Podczas szukania informacji na temat zagadnienia zapytań AXFR w kontekście cyberbezpieczeństwa natrafiono na projekt wykorzystujący tę samą podatność, jednak nastawiony na inne cele. Chodzi mianowicie o axfr-tools\cite{axfrtools} jednak autorzy projektu większy nacisk kładą na konkretną domenę i zbieranie historycznych zapisów, niż na analizowanie ogólnego trendu. Autorzy nie mieli również narzuconych ograniczeń czasowych skanowania, dlatego mogli pozwolić sobie na używanie narzędzi dostępnych w systemie linux (grep, sed, powłoka bash, produkcyjna wersja dig\'a) oraz na jednowątkowe generowanie zapytań. Warto dodać, że projekt był rozwijany w podobnym czasie co niniejsza praca magisterska, co może dowodzić temu, że temat cieszy się zainteresowaniem różnych osób (w tym przypadku pentesterów).

