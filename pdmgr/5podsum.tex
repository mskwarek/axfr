\section{Podsumowanie}

Podsumowując swoją pracę, chciałbym na pewno krytycznie na nią spojrzeć. Uważam, że zdecydowanie szybciej powinienem dojść do rozwiązania, które zostało ostatecznie wybrane. Zbyt dużo czasu zostało poświęconego na przedwczesną optymalizację. Ponadto, dużym problemem okazało się uruchomienie systemu na wydajniejszej maszynie. Pomimo dołożenia wszelkich starań do tego, aby system trafił już w pełni przetestowany i gotowy do działania nie udało się uniknąć problemów, które skróciły czas działania skanera. Mam tu głównie na myśli niepełne wykorzystanie dostępnych zasobów, które było widoczne dopiero na wydajnym komputerze.

Pomimo napotkanych problemów, myślę, że prace zostały wykonane w dużym stopniu. Z pewnością cieszy fakt, że można szukać ,,spektakularnych'' przypadków wykorzystania podatności AXFR na prawdziwych danych. 

To, co chciałbym zrealizować w najbliższej przyszłości to oczywiście dokładniejsze przestudiowanie zebranych informacji. Na razie skupiono się głównie na analizie wpisów TXT, co widać również w tym dokumencie. Byćmoże nie był to dobry kierunek, jednak te wpisy wydawały się najbardziej intrygujące oraz niepasujące do innych danych umieszczonych w pliku strefy (zone'y ?).